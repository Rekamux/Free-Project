\documentclass{beamer}

\usepackage[T1]{fontenc} 
\usepackage[utf8]{inputenc}
\usepackage[english]{babel}
%% \usepackage{pslatex}
%% \usepackage{colortbl} 
%% \usepackage{calc} 


\usetheme{Warsaw}

\definecolor{fondtitre}{rgb}{0.20,0.43,0.09}  % vert fonce
\definecolor{coultitre}{rgb}{0.40,0.05,0.05}  % marron
\definecolor{fondtexte}{rgb}{1,1,1}           % fond blanc
\definecolor{autre1}{RGB}{250,150,5}          % vieux mauve
\definecolor{autre2}{RGB}{235,175,235}        % horrible r
\colorlet{coultexte}{black}


\setbeamercolor{structure}{fg=coultitre, bg=fondtitre!40}
\setbeamercolor{block body}{bg=fondtexte}
\setbeamercolor{normal text}{fg=coultexte,bg=fondtexte}


\setbeamertemplate{footline}{
  \hbox{
    \hspace*{-0.06cm}

    \begin{beamercolorbox}[wd=.3\paperwidth,ht=2.25ex,dp=1ex,center]{title in head/foot}%
      \usebeamerfont{author in head/foot}\insertshortauthor
    \end{beamercolorbox}%

    \begin{beamercolorbox}[wd=.4\paperwidth,ht=2.25ex,dp=1ex,center]{title in head/foot}%
      \usebeamerfont{title in head/foot}\insertshorttitle
    \end{beamercolorbox}%

    \begin{beamercolorbox}[wd=.1\paperwidth,ht=2.25ex,dp=1ex,center]{date in head/foot}%
      \usebeamerfont{date in head/foot}
      \insertframenumber{} / \inserttotalframenumber\hspace*{2ex} 
    \end{beamercolorbox}%

    \begin{beamercolorbox}[wd=.2\paperwidth,ht=2.25ex,dp=1ex,center]{date in head/foot}%
      \usebeamerfont{date in head/foot}\insertdate
  \end{beamercolorbox}}%

  \vskip0pt%
}


\title[Free Project]{Sequence Complexity}
\author{Axel Schumacher}
\institute{Télécom Paristech}
\date{July $4^{th}$, 2011}


%---------------------------------------
%---------------------------------------


\begin{document}

\begin{frame}
  \titlepage
\end{frame}

\begin{frame}{Presentation}
  \begin{itemize}
    \item{1 2 3 4 ... ?}
    \item{9 9 9 ... ?}
    \item{1 2 2 3 3 3 ... ?}
    \item{1 2 2 3 3 4 ... ?}
    \item{1 1 2 1 2 3 1 1 2 1 2 3 1 2 3 4 ... ?}
    \item{1 1 1 1 2 2 ... ? }
  \end{itemize}
  \begin{center}
  Completing those sequences: easy for a human being\\
  (trivial notions in math)\\
  What about computers?
  \end{center}
\end{frame}

\begin{frame}{Constraints}
  \begin{itemize}
    \item{
      Two operators
      \begin{itemize}
        \item{Copy: C}
        \item{Increment: I}
      \end{itemize}}
    \item{No combinatory explosion}
    \item{Work and think in a "human" way}
  \end{itemize}
\end{frame}

\begin{frame}{Implementation}
  Suffixed notation, close to Factor:\\
  \begin{itemize}
    \item{
      Copy:\\
      \textit{What Times C}\\
      Examples:
      \begin{itemize}
        \item{5 5 5 5 $\rightarrow$ 5 4 \textit{C}}
        \item{1 2 1 2 1 2 $\rightarrow$ \{1 2\} 3 \textit{C}}
      \end{itemize}}
    \item{
      Increment:\\
      \textit{What Where Times I}\\
      Examples:
      \begin{itemize}
        \item{1 2 3 4 $\rightarrow$ 1 0 4 \textit{I}}
        \item{1 1 1 2 1 2 3 1 3 $\rightarrow$ \{1 1 1\} \{0 2\} 3 \textit{I}}
      \end{itemize}}
  \end{itemize}
\end{frame}

\begin{frame}{Extend a sequence}
  \begin{center}
    If we can compress a sequence to one operator and its arguments,\\
    we just have to increment its \textit{Times} field and decompress
  \end{center}
\end{frame}

\begin{frame}{Search order}
  We try:
  \begin{itemize}
    \item{to remove from 0 up to (length - 2) elements}
    \item{an operator}
    \item{a tested size}
  \end{itemize}
  With all those choices, we try to compress the whole sequence.\\
  We try another one if complexity is too high.
\end{frame}

\begin{frame}{Possible improvements}
  \begin{center}
    \begin{itemize}
    \item{Solve "once applied operators" problem}
    \item{Find a Kantian model}
    \item{Respect "human" memory use}
  \end{itemize}
   
  \end{center}
\end{frame}

\begin{frame}{Conclusion}
  \begin{center}
  Subject not that easy\\
  Factor is a powerful language\\
  Very enlightening project
  \end{center}
\end{frame}
 
\end{document}
