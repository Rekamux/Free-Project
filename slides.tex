\documentclass{beamer}

\usepackage[T1]{fontenc} 
\usepackage[utf8]{inputenc}
\usepackage[frenchb]{babel}
%% \usepackage{pslatex}
%% \usepackage{colortbl} 
%% \usepackage{calc} 


\usetheme{Warsaw}

\definecolor{fondtitre}{rgb}{0.20,0.43,0.09}  % vert fonce
\definecolor{coultitre}{rgb}{0.40,0.05,0.05}  % marron
\definecolor{fondtexte}{rgb}{1,1,1}           % fond blanc
\definecolor{autre1}{RGB}{250,150,5}          % vieux mauve
\definecolor{autre2}{RGB}{235,175,235}        % horrible r
\colorlet{coultexte}{black}


\setbeamercolor{structure}{fg=coultitre, bg=fondtitre!40}
\setbeamercolor{block body}{bg=fondtexte}
\setbeamercolor{normal text}{fg=coultexte,bg=fondtexte}


\setbeamertemplate{footline}{
  \hbox{
    \hspace*{-0.06cm}

    \begin{beamercolorbox}[wd=.3\paperwidth,ht=2.25ex,dp=1ex,center]{title in head/foot}%
      \usebeamerfont{author in head/foot}\insertshortauthor
    \end{beamercolorbox}%

    \begin{beamercolorbox}[wd=.4\paperwidth,ht=2.25ex,dp=1ex,center]{title in head/foot}%
      \usebeamerfont{title in head/foot}\insertshorttitle
    \end{beamercolorbox}%

    \begin{beamercolorbox}[wd=.1\paperwidth,ht=2.25ex,dp=1ex,center]{date in head/foot}%
      \usebeamerfont{date in head/foot}
      \insertframenumber{} / \inserttotalframenumber\hspace*{2ex} 
    \end{beamercolorbox}%

    \begin{beamercolorbox}[wd=.2\paperwidth,ht=2.25ex,dp=1ex,center]{date in head/foot}%
      \usebeamerfont{date in head/foot}\insertdate
  \end{beamercolorbox}}%

  \vskip0pt%
}


\title[PLNC]{Extension de listes}
\author{Axel Schumacher}
\institute{Télécom Paristech}
\date{27 juin 2011}


%---------------------------------------
%---------------------------------------


\begin{document}

\begin{frame}
  \titlepage
\end{frame}

\begin{frame}{Présentation}
  \begin{itemize}
    \item{1 2 3 4 ... ?}
    \item{9 9 9 ... ?}
    \item{1 2 2 3 3 3 ... ?}
    \item{1 2 2 3 3 4 ... ?}
    \item{1 1 2 1 2 3 1 1 2 1 2 3 1 2 3 4 ... ?}
    \item{1 1 1 1 2 2 ... ? }
  \end{itemize}
  \begin{center}
  Compléter ces listes est très facile pour un être humain à partir de l'âge de 8 ans.\\
  Qu'en est-il de l'ordinateur ?
  \end{center}
\end{frame}

\begin{frame}{Contraintes}
  \begin{itemize}
    \item{
      Deux opérateurs
      \begin{itemize}
        \item{Copie : C}
        \item{Incrément : I}
      \end{itemize}}
    \item{Pas d'explosion combinatoire}
    \item{Une façon de fonctionner 'humaine'}
  \end{itemize}
\end{frame}

\begin{frame}{Résolution}
  Notation suffixée, proche du factor :\\
  \begin{itemize}
    \item{
      Copie :\\
      Quoi Combien C\\
      Exemples :
      \begin{itemize}
        \item{5 5 5 5 -> 5 4 C}
        \item{1 2 1 2 1 2 -> \{1 2\} 3 C}
      \end{itemize}}
    \item{
      Incrément :\\
      Quoi Où Combien I\\
      Exemples :
      \begin{itemize}
        \item{1 2 3 4 -> 1 0 4 I}
        \item{1 1 1 2 1 2 3 1 3 -> \{1 1 1\} \{0 2\} 3 I}
      \end{itemize}}
  \end{itemize}
\end{frame}

\begin{frame}{Étendre une liste}
  \begin{center}
  Si on réussit à résumer une liste à un opérateur, il suffit d'incrémenter son nombre d'applications.
  \end{center}
\end{frame}

\begin{frame}{Ordre de recherche}
  On utilise des amb sur :
  \begin{itemize}
    \item{Opérateur}
    \item{Taille}
    \item{Répéter sur une liste}
  \end{itemize}
  On fail en cas de complexité trop élevée.
\end{frame}

\begin{frame}{Amélioration}
  \begin{center}
  On rajoute un amb sur quoi enlever.\\
  Si en étendant on contient la liste initiale, on a gagné !\\Sinon on fail.\\

  Passons à présent à une petite démonstration...
  \end{center}
\end{frame}

  
\end{document}
