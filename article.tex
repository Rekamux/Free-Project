\documentclass[a4paper]{article}
\usepackage[latin1]{inputenc}
\usepackage[english]{babel}
\usepackage[usenames,dvipsnames]{pstricks}
\usepackage{epsfig}
\usepackage{amsfonts}
\usepackage{amsmath}
\usepackage{graphicx}
\usepackage[nottoc, notlof, notlot]{tocbibind}
\usepackage{hyperref}

\begin{document}

\title{Sequences complexity}
\author{Axel Schumacher\\T�l�com ParisTech\\Free Project}
\date{\today}

\maketitle

\newpage

\tableofcontents

\newpage

\section{Introduction}

As a student at T�l�com ParisTech, I chose to follow the 'Non-classic languages paradigms' course with Samuel Tardieu\cite{samuel}, while working on a free project with Jean-Louis Dessalles\cite{jld}.
Mr. Dessalles suggested a project based on a Komolgorov's complexity problem: given a small sequence of digits, shapes or characters, I had to design a small program able to complete this list as well as an eight-years old children. \cite{starting_point}

The main purpose of this project is to apply Kolmogorov theory on the given list in order to compress it in a optimal and human way, so that we will then be able to extend that list using found compression process.
On the 'Non-classic languages paradigms' course, we learned small spread languages as Haskel, Scala or Factor, and we were also supposed to perform a small implementation project using one of those languages.

Factor\cite{factor} is a quite memory-close language, able to perform back-tracking as well as Prolog using continuations and working on a stack, like a logical sequence.
Its powerful and non-standard architecture make him a very interesting language to work on, and that is why I chose to use it to implement my project.

\section{Problematic}

Humans' capability to compress and understand structures always fascinated computer sciences researchers.
Humans are able to perfectly perform tasks as recognizing a person's face, understand a language or, in our scope of interest, understand and complete a list of digits, alphabetics or shapes. Those tasks are very difficult to efficiently complete for machines, even using the most powerful computers.
That is why a huge domain of research still exists in that field, always in the same goal: give machines more and more of our 'human intelligence'.

\section{Basis}

We are now to try to light up some darks areas of this problem.
Let us imagine a very simple sequence:
$$1\ 2\ 2\ 3\ 3\ 3$$
Any eight-years old children would easily suggest that a good continuation of this sequence would be:
$$1\ 2\ 2\ 3\ 3\ 3\ 4\ 4\ 4\ 4$$
But how can one be so sure about it?
This is where we can introduce the notion of 'complexity', and more especially Kolmogorov's one:

"In algorithmic information theory (a subfield of computer science), the Kolmogorov complexity of an object, such as a piece of text, is a measure of the computational resources needed to specify the object." \cite{wikipedia_kolmogorov}

In other words, this complexity has the particularity to highly depends of using person, and is therefore really difficult to use inside a program.

Using this complexity, one could define $1\ 2\ 2\ 3\ 3\ 3$ as 'count from one and repeat each number as much as itself', which is really easier to understand, memorize and extend for an human than a random sequence, like a telephone or a credit card number. We will explore different ways to represent and manipulate such informations.

An other way to compute a list's complexity would be to use \textit{short term memory}, a list or tree containing last used elements.
An access to one of those elements would be its address for example binary coded.
This model is easily explainable considering that a human, if he was able to solve a problem using a specific operator, will naturally try it first when confronted to a new problem.
This list is implemented but not used because of its unpredictable character.

That is why in present version a standard \textit{binary coding} is used to compute a list's complexity, with 0 costing 0.

Given a way to compute complexity, one can try to compress a list using \textit{operators} while resulting complexity is lower than initial sequence's one.

\section{Tentatives}

\subsection{Three operators}

At first, I thought about using Factor's \textit{tuples}, special sequences of fields easily sharable between descendants, like standard \textit{Object Oriented Programming}.

I defined a tuple \textit{operator} containing a field named \textit{times}.
Operator had the following methods, or \textit{generic functions} in Factor:
\begin{itemize}
  \item{cost, a function returning a operator's cost regarding its fields values. At this time, I wasn't thinking about short term memory.}
  \item{apply, a function which uses a digit and generate a sequence by uncompressing this operator on it.}
  \item{search, a function which tries to find current operator's application on a sequence.}
\end{itemize}

Given those methods, I defined three operators:
\begin{itemize}
  \item{A copy operator: using its predecessor's \textit{times}, copies given argument \textit{times} }
  \item{An increment operator: increments and copy by one given argument \textit{times}. If this argument is an operator, it will copy it by incrementing its \textit{times} field. }
  \item{A step operator: contains also a \textit{gap} argument and an \textit{operator} field, applies its operator by skipping as much elements as given by \textit{gap} on a list.}
\end{itemize}

The step operator was useful for this case:
$$1\  2\  2\  3\  3\  3\  4\  4\  4\  4$$
When compressed once using copy operator, it gave:
$$(copy\ times(1))\ 1\ (copy\ times(2))\ 2\ (copy\ times(3))\ 3\ (copy\ times(4))\ 4$$
Here, we can easily see that we have an increment on the digits and on the operators, but separated by one element each time. Using step operator, it gives:
$$(step\ operator\ times(4)\ gap(1)\ operator(increment\ times(1)))\ 1\ 2\ 3\ 4$$
The step operator will apply three times with a gap of one. As between each gap we only have one element, the increment operator in \textit{operator} field only applies once each time. The step operator will apply on the directly following element, the copy operator.
A final compression would give:
$$(step\ operator\ times(4)\ gap(1)\ operator(increment\ times(1)))\ (copy\ times(4))\ 4$$

Speaking of complexity, if we count 1 for each operator plus recursive cost of all its arguments, the sequence $1\ 2\ 2\ 3\ 3\ 3\ 4\ 4\ 4\ 4$ will cost 23, where the compressed version costs 14. Hence, we can say that we have a compressed version.

Copy and increment search were easy, I just had to compare current result with the next sequence's element. But for the step-operator, how can one determine the gap before searching? I tried each operator but with a gap of one. The problem was that it didn't used all the power of the step operator, but the search would cost a lot in an other way. And what a step operator applied on a step operator would mean? A real complex, not useful nor handy object actually.

This solution displeased Jean-Louis Dessalles, because of this strange step-operator not really 'natural', and difficult to understand and to use. That is why he asked me to only use two operators: copy and increment.

\section{Current solution}

\section{Ways to continue}

\newpage

\begin{thebibliography}{9}

\bibitem{samuel} Samuel Tardieu's website: \url{http://www.rfc1149.net/}
\bibitem{jld} Jean Louis Dessalles' website: \url{http://perso.telecom-paristech.fr/~jld/}
\bibitem{starting_point} A starting point: \url{http://icc.enst.fr/PLC/Learn.html}, section Complexity
\bibitem{factor} Factor language: \url{http://factorcode.org/}
\bibitem{wikipedia_kolmogorov} Kolmogorov's Complexity on Wikiedia: \url{http://en.wikipedia.org/wiki/Kolmogorov_complexity}

\end{thebibliography}

\end{document}
