\documentclass[a4paper]{article}
\usepackage[latin1]{inputenc}
\usepackage[english]{babel}
\usepackage[usenames,dvipsnames]{pstricks}
\usepackage{epsfig}
\usepackage{amsfonts}
\usepackage{amsmath}
\usepackage{graphicx}
\usepackage[nottoc, notlof, notlot]{tocbibind}

\begin{document}

\title{Sequences complexity}
\author{Axel Schumacher\\T�l�com ParisTech\\Free Project}
\date{\today}

\maketitle

\newpage

\tableofcontents

\newpage

\section{Introduction}

As a student at T�l�com ParisTech, I chose to follow the 'Non-classic languages paradigms' course with Samuel Tardieu, while working on a free project with Jean-Louis Dessalles.
Mr. Dessalles suggested a project based on a Komolgorov's complexity problem: given a small sequence of digits, shapes or characters, I had to design a small program able to complete this list as well as an eight-years old children.
The main purpose of this project is to apply Kolmogorov theory on the given list in order to compress it in a optimal and human way, so that we will then be able to extend that list using found compression process.
On the 'Non-classic languages paradigms' course, we learned small spread languages as Haskel, Scala or Factor, and we were also supposed to perform a small implementation project using one of those languages.
Factor is a quite memory-close language, able to perform back-tracking as well as Prolog using continuations.
Its powerful and non-standard architecture make him a very interesting language to work on.

\section{Problematic}

Humans' capability to compress and understand structures always fascinated computer sciences researchers.
Humans are able to perfectly perform tasks as recognizing a person's face, understand a language or, in our scope of interest, understand and complete a list of digits, alphabetics or shapes. Those tasks are very difficult to efficiently complete for machines, even using the most powerful computers.
That is why a huge domain of research still exists in this field, always in the same goal: give machines more and more of our 'human intelligence'.

\section{Basis}

Were now to try to light up some darks areas of this problem.
Let us imagine a very simple sequence:
$$1 2 2 3 3 3$$
Any eight-years old children would easily suggest that a good continuation of this sequence would be:
$$1 2 2 3 3 3 4 4 4 4$$
But how can one be so sure about it?
This is where we can introduce the notion of 'complexity', and more especially Kolmogorov's one:


\begin{thebibliography}{1}

\bibitem{A starting point} http://icc.enst.fr/PLC/Learn.html, section Complexity

\end{thebibliography}

\end{document}
